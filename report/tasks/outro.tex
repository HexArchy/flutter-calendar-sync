\chapter*{Заключение}
\addcontentsline{toc}{chapter}{Заключение}

В ходе выполнения данной лабораторной работы была успешно разработана и проанализирована система синхронизации календаря для студентов ИТМО, состоящая из backend-сервиса на языке Go и мобильного приложения на Flutter.

\textbf{Основные результаты работы:}

\begin{enumerate}
    \item \textbf{Backend-сервис itmo-calendar} --- разработан полнофункциональный REST API сервис с использованием современной use-case driven архитектуры. Сервис обеспечивает получение расписания студентов, генерацию iCalendar файлов и подписку пользователей. Реализованы механизмы аутентификации через JWT и поддержка HTTPS для безопасной передачи данных.
    
    \item \textbf{Flutter-приложение calendar-sync} --- создано кроссплатформенное мобильное приложение с современным пользовательским интерфейсом. Приложение использует архитектуру с разделением ответственности, включающую сервисы для работы с API и календарем, провайдеры для управления состоянием и модульную структуру виджетов.
    
    \item \textbf{Механизмы безопасности} --- успешно реализована защита от создания скриншотов с использованием пакета \texttt{screen\_protector}, что предотвращает несанкционированное сохранение содержимого экрана приложения.
    
    \item \textbf{Сетевое взаимодействие} --- обеспечено корректное взаимодействие мобильного приложения с backend API, включая обработку самоподписанных SSL-сертификатов для разработки и настройку HTTP-клиента с таймаутами.
    
    \item \textbf{Анализ безопасности} --- проведен комплексный анализ безопасности разработанного приложения, включающий:
    \begin{itemize}
        \item Сборку APK-файлов с различными настройками обфускации
        \item Декомпиляцию и сравнительный анализ полученных файлов
        \item Сканирование утилитой \texttt{apkleaks} для поиска потенциальных утечек данных
        \item Проверку через сервис VirusTotal на наличие вредоносного кода
    \end{itemize}
    
    \item \textbf{Функциональность приложения} --- реализованы все основные функции календарного приложения: отображение расписания в различных форматах (месяц, неделя, список), настройки синхронизации, генерация ссылок для импорта календаря в сторонние приложения.
\end{enumerate}

\textbf{Выводы по анализу безопасности:}

Проведенный анализ показал, что разработанное приложение соответствует базовым требованиям безопасности. Обфускация кода эффективно усложняет процесс реверс-инжиниринга, а защита от скриншотов предотвращает несанкционированное сохранение конфиденциальной информации. Сканирование специализированными инструментами не выявило критических уязвимостей.
